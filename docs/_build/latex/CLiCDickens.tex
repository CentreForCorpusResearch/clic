% Generated by Sphinx.
\def\sphinxdocclass{report}
\documentclass[letterpaper,10pt,english]{sphinxmanual}
\usepackage[utf8]{inputenc}
\DeclareUnicodeCharacter{00A0}{\nobreakspace}
\usepackage{cmap}
\usepackage[T1]{fontenc}
\usepackage{babel}
\usepackage{times}
\usepackage[Bjarne]{fncychap}
\usepackage{longtable}
\usepackage{sphinx}
\usepackage{multirow}
\usepackage{eqparbox}


\addto\captionsenglish{\renewcommand{\figurename}{Fig. }}
\addto\captionsenglish{\renewcommand{\tablename}{Table }}
\SetupFloatingEnvironment{literal-block}{name=Listing }



\title{CLiC Dickens Documentation}
\date{March 15, 2016}
\release{1.3}
\author{J. de Joode}
\newcommand{\sphinxlogo}{}
\renewcommand{\releasename}{Release}
\setcounter{tocdepth}{2}
\makeindex

\makeatletter
\def\PYG@reset{\let\PYG@it=\relax \let\PYG@bf=\relax%
    \let\PYG@ul=\relax \let\PYG@tc=\relax%
    \let\PYG@bc=\relax \let\PYG@ff=\relax}
\def\PYG@tok#1{\csname PYG@tok@#1\endcsname}
\def\PYG@toks#1+{\ifx\relax#1\empty\else%
    \PYG@tok{#1}\expandafter\PYG@toks\fi}
\def\PYG@do#1{\PYG@bc{\PYG@tc{\PYG@ul{%
    \PYG@it{\PYG@bf{\PYG@ff{#1}}}}}}}
\def\PYG#1#2{\PYG@reset\PYG@toks#1+\relax+\PYG@do{#2}}

\expandafter\def\csname PYG@tok@gd\endcsname{\def\PYG@tc##1{\textcolor[rgb]{0.63,0.00,0.00}{##1}}}
\expandafter\def\csname PYG@tok@gu\endcsname{\let\PYG@bf=\textbf\def\PYG@tc##1{\textcolor[rgb]{0.50,0.00,0.50}{##1}}}
\expandafter\def\csname PYG@tok@gt\endcsname{\def\PYG@tc##1{\textcolor[rgb]{0.00,0.27,0.87}{##1}}}
\expandafter\def\csname PYG@tok@gs\endcsname{\let\PYG@bf=\textbf}
\expandafter\def\csname PYG@tok@gr\endcsname{\def\PYG@tc##1{\textcolor[rgb]{1.00,0.00,0.00}{##1}}}
\expandafter\def\csname PYG@tok@cm\endcsname{\let\PYG@it=\textit\def\PYG@tc##1{\textcolor[rgb]{0.25,0.50,0.56}{##1}}}
\expandafter\def\csname PYG@tok@vg\endcsname{\def\PYG@tc##1{\textcolor[rgb]{0.73,0.38,0.84}{##1}}}
\expandafter\def\csname PYG@tok@vi\endcsname{\def\PYG@tc##1{\textcolor[rgb]{0.73,0.38,0.84}{##1}}}
\expandafter\def\csname PYG@tok@mh\endcsname{\def\PYG@tc##1{\textcolor[rgb]{0.13,0.50,0.31}{##1}}}
\expandafter\def\csname PYG@tok@cs\endcsname{\def\PYG@tc##1{\textcolor[rgb]{0.25,0.50,0.56}{##1}}\def\PYG@bc##1{\setlength{\fboxsep}{0pt}\colorbox[rgb]{1.00,0.94,0.94}{\strut ##1}}}
\expandafter\def\csname PYG@tok@ge\endcsname{\let\PYG@it=\textit}
\expandafter\def\csname PYG@tok@vc\endcsname{\def\PYG@tc##1{\textcolor[rgb]{0.73,0.38,0.84}{##1}}}
\expandafter\def\csname PYG@tok@il\endcsname{\def\PYG@tc##1{\textcolor[rgb]{0.13,0.50,0.31}{##1}}}
\expandafter\def\csname PYG@tok@go\endcsname{\def\PYG@tc##1{\textcolor[rgb]{0.20,0.20,0.20}{##1}}}
\expandafter\def\csname PYG@tok@cp\endcsname{\def\PYG@tc##1{\textcolor[rgb]{0.00,0.44,0.13}{##1}}}
\expandafter\def\csname PYG@tok@gi\endcsname{\def\PYG@tc##1{\textcolor[rgb]{0.00,0.63,0.00}{##1}}}
\expandafter\def\csname PYG@tok@gh\endcsname{\let\PYG@bf=\textbf\def\PYG@tc##1{\textcolor[rgb]{0.00,0.00,0.50}{##1}}}
\expandafter\def\csname PYG@tok@ni\endcsname{\let\PYG@bf=\textbf\def\PYG@tc##1{\textcolor[rgb]{0.84,0.33,0.22}{##1}}}
\expandafter\def\csname PYG@tok@nl\endcsname{\let\PYG@bf=\textbf\def\PYG@tc##1{\textcolor[rgb]{0.00,0.13,0.44}{##1}}}
\expandafter\def\csname PYG@tok@nn\endcsname{\let\PYG@bf=\textbf\def\PYG@tc##1{\textcolor[rgb]{0.05,0.52,0.71}{##1}}}
\expandafter\def\csname PYG@tok@no\endcsname{\def\PYG@tc##1{\textcolor[rgb]{0.38,0.68,0.84}{##1}}}
\expandafter\def\csname PYG@tok@na\endcsname{\def\PYG@tc##1{\textcolor[rgb]{0.25,0.44,0.63}{##1}}}
\expandafter\def\csname PYG@tok@nb\endcsname{\def\PYG@tc##1{\textcolor[rgb]{0.00,0.44,0.13}{##1}}}
\expandafter\def\csname PYG@tok@nc\endcsname{\let\PYG@bf=\textbf\def\PYG@tc##1{\textcolor[rgb]{0.05,0.52,0.71}{##1}}}
\expandafter\def\csname PYG@tok@nd\endcsname{\let\PYG@bf=\textbf\def\PYG@tc##1{\textcolor[rgb]{0.33,0.33,0.33}{##1}}}
\expandafter\def\csname PYG@tok@ne\endcsname{\def\PYG@tc##1{\textcolor[rgb]{0.00,0.44,0.13}{##1}}}
\expandafter\def\csname PYG@tok@nf\endcsname{\def\PYG@tc##1{\textcolor[rgb]{0.02,0.16,0.49}{##1}}}
\expandafter\def\csname PYG@tok@si\endcsname{\let\PYG@it=\textit\def\PYG@tc##1{\textcolor[rgb]{0.44,0.63,0.82}{##1}}}
\expandafter\def\csname PYG@tok@s2\endcsname{\def\PYG@tc##1{\textcolor[rgb]{0.25,0.44,0.63}{##1}}}
\expandafter\def\csname PYG@tok@nt\endcsname{\let\PYG@bf=\textbf\def\PYG@tc##1{\textcolor[rgb]{0.02,0.16,0.45}{##1}}}
\expandafter\def\csname PYG@tok@nv\endcsname{\def\PYG@tc##1{\textcolor[rgb]{0.73,0.38,0.84}{##1}}}
\expandafter\def\csname PYG@tok@s1\endcsname{\def\PYG@tc##1{\textcolor[rgb]{0.25,0.44,0.63}{##1}}}
\expandafter\def\csname PYG@tok@ch\endcsname{\let\PYG@it=\textit\def\PYG@tc##1{\textcolor[rgb]{0.25,0.50,0.56}{##1}}}
\expandafter\def\csname PYG@tok@m\endcsname{\def\PYG@tc##1{\textcolor[rgb]{0.13,0.50,0.31}{##1}}}
\expandafter\def\csname PYG@tok@gp\endcsname{\let\PYG@bf=\textbf\def\PYG@tc##1{\textcolor[rgb]{0.78,0.36,0.04}{##1}}}
\expandafter\def\csname PYG@tok@sh\endcsname{\def\PYG@tc##1{\textcolor[rgb]{0.25,0.44,0.63}{##1}}}
\expandafter\def\csname PYG@tok@ow\endcsname{\let\PYG@bf=\textbf\def\PYG@tc##1{\textcolor[rgb]{0.00,0.44,0.13}{##1}}}
\expandafter\def\csname PYG@tok@sx\endcsname{\def\PYG@tc##1{\textcolor[rgb]{0.78,0.36,0.04}{##1}}}
\expandafter\def\csname PYG@tok@bp\endcsname{\def\PYG@tc##1{\textcolor[rgb]{0.00,0.44,0.13}{##1}}}
\expandafter\def\csname PYG@tok@c1\endcsname{\let\PYG@it=\textit\def\PYG@tc##1{\textcolor[rgb]{0.25,0.50,0.56}{##1}}}
\expandafter\def\csname PYG@tok@o\endcsname{\def\PYG@tc##1{\textcolor[rgb]{0.40,0.40,0.40}{##1}}}
\expandafter\def\csname PYG@tok@kc\endcsname{\let\PYG@bf=\textbf\def\PYG@tc##1{\textcolor[rgb]{0.00,0.44,0.13}{##1}}}
\expandafter\def\csname PYG@tok@c\endcsname{\let\PYG@it=\textit\def\PYG@tc##1{\textcolor[rgb]{0.25,0.50,0.56}{##1}}}
\expandafter\def\csname PYG@tok@mf\endcsname{\def\PYG@tc##1{\textcolor[rgb]{0.13,0.50,0.31}{##1}}}
\expandafter\def\csname PYG@tok@err\endcsname{\def\PYG@bc##1{\setlength{\fboxsep}{0pt}\fcolorbox[rgb]{1.00,0.00,0.00}{1,1,1}{\strut ##1}}}
\expandafter\def\csname PYG@tok@mb\endcsname{\def\PYG@tc##1{\textcolor[rgb]{0.13,0.50,0.31}{##1}}}
\expandafter\def\csname PYG@tok@ss\endcsname{\def\PYG@tc##1{\textcolor[rgb]{0.32,0.47,0.09}{##1}}}
\expandafter\def\csname PYG@tok@sr\endcsname{\def\PYG@tc##1{\textcolor[rgb]{0.14,0.33,0.53}{##1}}}
\expandafter\def\csname PYG@tok@mo\endcsname{\def\PYG@tc##1{\textcolor[rgb]{0.13,0.50,0.31}{##1}}}
\expandafter\def\csname PYG@tok@kd\endcsname{\let\PYG@bf=\textbf\def\PYG@tc##1{\textcolor[rgb]{0.00,0.44,0.13}{##1}}}
\expandafter\def\csname PYG@tok@mi\endcsname{\def\PYG@tc##1{\textcolor[rgb]{0.13,0.50,0.31}{##1}}}
\expandafter\def\csname PYG@tok@kn\endcsname{\let\PYG@bf=\textbf\def\PYG@tc##1{\textcolor[rgb]{0.00,0.44,0.13}{##1}}}
\expandafter\def\csname PYG@tok@cpf\endcsname{\let\PYG@it=\textit\def\PYG@tc##1{\textcolor[rgb]{0.25,0.50,0.56}{##1}}}
\expandafter\def\csname PYG@tok@kr\endcsname{\let\PYG@bf=\textbf\def\PYG@tc##1{\textcolor[rgb]{0.00,0.44,0.13}{##1}}}
\expandafter\def\csname PYG@tok@s\endcsname{\def\PYG@tc##1{\textcolor[rgb]{0.25,0.44,0.63}{##1}}}
\expandafter\def\csname PYG@tok@kp\endcsname{\def\PYG@tc##1{\textcolor[rgb]{0.00,0.44,0.13}{##1}}}
\expandafter\def\csname PYG@tok@w\endcsname{\def\PYG@tc##1{\textcolor[rgb]{0.73,0.73,0.73}{##1}}}
\expandafter\def\csname PYG@tok@kt\endcsname{\def\PYG@tc##1{\textcolor[rgb]{0.56,0.13,0.00}{##1}}}
\expandafter\def\csname PYG@tok@sc\endcsname{\def\PYG@tc##1{\textcolor[rgb]{0.25,0.44,0.63}{##1}}}
\expandafter\def\csname PYG@tok@sb\endcsname{\def\PYG@tc##1{\textcolor[rgb]{0.25,0.44,0.63}{##1}}}
\expandafter\def\csname PYG@tok@k\endcsname{\let\PYG@bf=\textbf\def\PYG@tc##1{\textcolor[rgb]{0.00,0.44,0.13}{##1}}}
\expandafter\def\csname PYG@tok@se\endcsname{\let\PYG@bf=\textbf\def\PYG@tc##1{\textcolor[rgb]{0.25,0.44,0.63}{##1}}}
\expandafter\def\csname PYG@tok@sd\endcsname{\let\PYG@it=\textit\def\PYG@tc##1{\textcolor[rgb]{0.25,0.44,0.63}{##1}}}

\def\PYGZbs{\char`\\}
\def\PYGZus{\char`\_}
\def\PYGZob{\char`\{}
\def\PYGZcb{\char`\}}
\def\PYGZca{\char`\^}
\def\PYGZam{\char`\&}
\def\PYGZlt{\char`\<}
\def\PYGZgt{\char`\>}
\def\PYGZsh{\char`\#}
\def\PYGZpc{\char`\%}
\def\PYGZdl{\char`\$}
\def\PYGZhy{\char`\-}
\def\PYGZsq{\char`\'}
\def\PYGZdq{\char`\"}
\def\PYGZti{\char`\~}
% for compatibility with earlier versions
\def\PYGZat{@}
\def\PYGZlb{[}
\def\PYGZrb{]}
\makeatother

\renewcommand\PYGZsq{\textquotesingle}

\begin{document}

\maketitle
\tableofcontents
\phantomsection\label{index::doc}


Contents:


\chapter{CLiC for end-users}
\label{endusers::doc}\label{endusers:clic-for-end-users}\label{endusers:welcome-to-clic-dickens-s-documentation}

\section{Definitions}
\label{endusers:definitions}
\includegraphics{{definition}.png}


\section{The underlying annotation}
\label{endusers:the-underlying-annotation}

\section{The corpora}
\label{endusers:the-corpora}

\section{How to use ...}
\label{endusers:how-to-use}

\subsection{the concordance}
\label{endusers:the-concordance}

\subsection{clusters}
\label{endusers:clusters}

\subsection{keywords}
\label{endusers:keywords}

\subsection{subsets}
\label{endusers:subsets}

\subsection{patterns}
\label{endusers:patterns}

\subsection{user annotation}
\label{endusers:user-annotation}

\chapter{CLiC for administrators}
\label{admin::doc}\label{admin:clic-for-administrators}

\section{Installing CLiC on your own server}
\label{admin:installing-clic-on-your-own-server}
For the deployment of CLiC we heavily rely on Docker. This has several benefits:
it packages all the dependencies of CLiC together in a simple image and it makes
a deployment much faster and possible on many different platforms.


\subsection{Install Docker on a vanilla Ubuntu server}
\label{admin:install-docker-on-a-vanilla-ubuntu-server}
\begin{Verbatim}[commandchars=\\\{\}]
\PYGZsh{} Install Docker in a way that can easily be upgraded
sudo apt\PYGZhy{}get update

sudo apt\PYGZhy{}key adv \PYGZhy{}\PYGZhy{}keyserver hkp://pgp.mit.edu:80 \PYGZhy{}\PYGZhy{}recv\PYGZhy{}keys
58118e89f3a912897c070adbf76221572c52609d

\PYGZsh{} open the /etc/apt/sources.list.d/docker.list file in your favorite
editor. if the file doesn’t exist, create it. and add:

\PYGZsh{} this is not a command, but something that needs to be pasted in the
file opened

deb https://apt.dockerproject.org/repo ubuntu\PYGZhy{}trusty main

sudo apt\PYGZhy{}get update

\PYGZsh{} verify that apt is pulling from the right repository.

sudo apt\PYGZhy{}cache policy docker\PYGZhy{}engine

sudo apt\PYGZhy{}get install linux\PYGZhy{}image\PYGZhy{}generic\PYGZhy{}lts\PYGZhy{}trusty

\PYGZsh{} you need to reboot, this can cause some issues where the firewall
settings would not be saved, this was solved by loading the firewall
explicitly at the restart

sudo reboot

sudo apt\PYGZhy{}get update

sudo apt\PYGZhy{}get install docker\PYGZhy{}engine

sudo docker run hello\PYGZhy{}world
\end{Verbatim}


\subsection{Configure Docker}
\label{admin:configure-docker}
\begin{Verbatim}[commandchars=\\\{\}]
\PYGZsh{} set UFW’s forwarding policy appropriately.
\PYGZsh{} Open /etc/default/ufw file for editing.
sudo vimsu /etc/default/ufw

\PYGZsh{} Set the DEFAULT\PYGZbs{}\PYGZus{}FORWARD\PYGZbs{}\PYGZus{}POLICY policy to:
DEFAULT\PYGZbs{}\PYGZus{}FORWARD\PYGZbs{}\PYGZus{}POLICY=\PYGZdq{}ACCEPT\PYGZdq{}

sudo ufw reload

\PYGZsh{}TODO check if this is persistent

\PYGZsh{} Allow incoming connections on the Docker port.
\PYGZdl{} sudo ufw allow 2375/tcp

\PYGZsh{} Configure Docker to start on boot
\PYGZsh{} DOCS SAY \PYGZdl{} sudo systemctl enable docker
\PYGZsh{} but I think:

sudo update\PYGZhy{}rc.d docker defaults
\end{Verbatim}


\subsection{Quick Docker command guide}
\label{admin:quick-docker-command-guide}
You might have to prepend \code{sudo} to the commands below, depending on your
environment.

\begin{Verbatim}[commandchars=\\\{\}]
\PYGZsh{} is docker installed
docker info

\PYGZsh{} activate docker
docker\PYGZhy{}machine start default
eval \PYGZdq{}\PYGZdl{}(docker\PYGZhy{}machine env default)\PYGZdq{}

\PYGZsh{} to find localhost ping
docker\PYGZhy{}machine env default

\PYGZsh{} to build
cd \PYGZti{}/ImagesDocker/clic\PYGZhy{}docker/
docker build \PYGZhy{}t jdejoode/clic:v0 .

\PYGZsh{} list images
docker images

\PYGZsh{} run a docker image
docker run \PYGZhy{}d \PYGZhy{}P \PYGZhy{}i \PYGZhy{}t \PYGZhy{}\PYGZhy{}name apache11 jdejoode/clic:v0 docker ps

\PYGZsh{} find info on container
docker port apache11
docker logs a5a665d32

\PYGZsh{} for live updates a la Flask
docker logs –f a6516a51sd f651

\PYGZsh{} stop all docker containers
docker stop \PYGZdl{}(docker ps \PYGZhy{}a \PYGZhy{}q)

\PYGZsh{} remove them
docker rm \PYGZdl{}(docker ps \PYGZhy{}a \PYGZhy{}\PYGZbs{} **q**\PYGZbs{} )

\PYGZsh{} ssh into container
docker exec \PYGZhy{}i \PYGZhy{}t fbd8112 bash

\PYGZsh{} command on actual deploy is slightly different for caching and
rounting purposes:
\PYGZsh{} \PYGZhy{}v /path/on/host:/path/in/container
docker run \PYGZhy{}p 80:8080 –v /tmp/cache:/tmp/cache …

docker run \PYGZhy{}d \PYGZhy{}P \PYGZhy{}\PYGZhy{}name clic2 \PYGZhy{}v
/bin:/clic\PYGZhy{}project/clic/dbs/dickens/indexes jdejoode/clic:v0

\PYGZsh{} to see what processes run in the container
docker top clic0

\PYGZsh{} remove untagged images
docker rmi \PYGZdl{}(docker images \PYGZbs{}\textbar{} grep \PYGZdq{}\PYGZca{}\PYGZlt{}none\PYGZgt{}\PYGZdq{} \PYGZbs{}\textbar{} awk \PYGZsq{}\PYGZob{}print \PYGZdl{}3\PYGZcb{}\PYGZsq{})

\PYGZsh{} deploy
\PYGZsh{} on macbook
\PYGZsh{} https://docs.docker.com/docker\PYGZhy{}hub/repos/
docker login
docker push jdejoode/clic:latest
\end{Verbatim}


\chapter{CLiC for developers}
\label{apidoc::doc}\label{apidoc:clic-for-developers}

\section{Cheshire3}
\label{apidoc:cheshire3}

\subsection{The data-model}
\label{apidoc:the-data-model}

\section{CLiC Concordance}
\label{apidoc:clic-concordance}\label{apidoc:module-concordance}\index{concordance (module)}
CLiC Concordance based on cheshire3 indexes.
\index{Concordance (class in concordance)}

\begin{fulllineitems}
\phantomsection\label{apidoc:concordance.Concordance}\pysigline{\strong{class }\code{concordance.}\bfcode{Concordance}}
This concordance takes terms, index names, book selections, and search type
as input values and returns json with the search term, ten words to the left and
ten to the right, and location information.

This can be used in an ajax api.
\index{build\_and\_run\_query() (concordance.Concordance method)}

\begin{fulllineitems}
\phantomsection\label{apidoc:concordance.Concordance.build_and_run_query}\pysiglinewithargsret{\bfcode{build\_and\_run\_query}}{\emph{terms}, \emph{idxName}, \emph{Materials}, \emph{selectWords}}{}
Builds a cheshire query and runs it.

Its output is a tuple of which the first element is a resultset and
the second element is number of search terms in the query.

\end{fulllineitems}

\index{create\_concordance() (concordance.Concordance method)}

\begin{fulllineitems}
\phantomsection\label{apidoc:concordance.Concordance.create_concordance}\pysiglinewithargsret{\bfcode{create\_concordance}}{\emph{terms}, \emph{idxName}, \emph{Materials}, \emph{selectWords}}{}
main concordance method
create a list of lists containing each three contexts left - node -right,
and a list within those contexts containing each word.
Add two separate lists containing metadata information:
{[}
{[}left context - word 1, word 2, etc.{]},
{[}node - word 1, word 2, etc{]},
{[}right context - word 1, etc{]},
{[}chapter metadata{]},
{[}book metadata{]}
{]},
etc.

\end{fulllineitems}


\end{fulllineitems}



\section{CLiC Clusters}
\label{apidoc:clic-clusters}\label{apidoc:module-clusters}\index{clusters (module)}
Tool to create wordlists based on the entries in an index.
\index{Clusters (class in clusters)}

\begin{fulllineitems}
\phantomsection\label{apidoc:clusters.Clusters}\pysigline{\strong{class }\code{clusters.}\bfcode{Clusters}}
Class that does all the heavy weighting. It makes the connection with
cheshire3, uses the input parameters (indexname and subcorpus/Materials) to
return a list of words and their total number of occurrences.

For instance,
\begin{quote}

\code{the 98021}

\code{to  78465}

...
\end{quote}
\begin{description}
\item[{or}] \leavevmode
\code{he said  8937}

\code{she said 6732}

...

\end{description}
\index{list\_clusters() (clusters.Clusters method)}

\begin{fulllineitems}
\phantomsection\label{apidoc:clusters.Clusters.list_clusters}\pysiglinewithargsret{\bfcode{list\_clusters}}{\emph{idxName}, \emph{Materials}}{}
Build a list of clusters and their occurrences.

Limit the list to the first 3000 entries.

\end{fulllineitems}


\end{fulllineitems}



\section{CLiC Keywords}
\label{apidoc:module-keywords}\label{apidoc:clic-keywords}\index{keywords (module)}
Module to compute keywords (words that are used significantly more frequently
in one corpus than they are in a reference corpus).
\index{Keywords (class in keywords)}

\begin{fulllineitems}
\phantomsection\label{apidoc:keywords.Keywords}\pysigline{\strong{class }\code{keywords.}\bfcode{Keywords}}
Class to compute keywords based on an test index (the corpus of analysis),
a reference index (the corpus of reference), and a P value.
\index{list\_keywords() (keywords.Keywords method)}

\begin{fulllineitems}
\phantomsection\label{apidoc:keywords.Keywords.list_keywords}\pysiglinewithargsret{\bfcode{list\_keywords}}{\emph{testIdxName}, \emph{testMaterials}, \emph{refIdxName}, \emph{refMaterials}, \emph{pValue}}{}
Return a sorted list of keywords. Limited to the first 5000 items.

\end{fulllineitems}


\end{fulllineitems}



\section{CLiC Chapter Repository}
\label{apidoc:clic-chapter-repository}\label{apidoc:module-chapter_repository}\index{chapter\_repository (module)}
Display the texts available in the cheshire3 database. Also highlight specific
items that were previously retrieved with a concordance.
\index{ChapterRepository (class in chapter\_repository)}

\begin{fulllineitems}
\phantomsection\label{apidoc:chapter_repository.ChapterRepository}\pysigline{\strong{class }\code{chapter\_repository.}\bfcode{ChapterRepository}}
Responsible for providing access to chapter resources within Cheshire.
\index{get\_book\_title() (chapter\_repository.ChapterRepository method)}

\begin{fulllineitems}
\phantomsection\label{apidoc:chapter_repository.ChapterRepository.get_book_title}\pysiglinewithargsret{\bfcode{get\_book\_title}}{\emph{book}}{}
Gets the title of a book from the json file booklist.json

book -- string - the book id/accronym e.g. BH

\end{fulllineitems}

\index{get\_chapter() (chapter\_repository.ChapterRepository method)}

\begin{fulllineitems}
\phantomsection\label{apidoc:chapter_repository.ChapterRepository.get_chapter}\pysiglinewithargsret{\bfcode{get\_chapter}}{\emph{chapter\_number}, \emph{book}}{}
Returns transformed XML for given chapter \& book

chapter\_number -- integer
book -- string - the book id/accronym e.g. BH

\end{fulllineitems}

\index{get\_chapter\_with\_highlighted\_search\_term() (chapter\_repository.ChapterRepository method)}

\begin{fulllineitems}
\phantomsection\label{apidoc:chapter_repository.ChapterRepository.get_chapter_with_highlighted_search_term}\pysiglinewithargsret{\bfcode{get\_chapter\_with\_highlighted\_search\_term}}{\emph{chapter\_number}, \emph{book}, \emph{wid}, \emph{search\_term}}{}
Returns transformed XML for given chapter \& book with the search
highlighted.

We create the transformer directly so that we can pass extra parameters
to it at runtime. In this case the search term.

chapter\_number -- integer
book -- string - the book id/accronym e.g. BH
wid -- integer - word index
search\_term -- string - term to highlight

\end{fulllineitems}

\index{get\_raw\_chapter() (chapter\_repository.ChapterRepository method)}

\begin{fulllineitems}
\phantomsection\label{apidoc:chapter_repository.ChapterRepository.get_raw_chapter}\pysiglinewithargsret{\bfcode{get\_raw\_chapter}}{\emph{chapter\_number}, \emph{book}}{}
Returns raw chapter XML for given chapter \& book

chapter\_number -- integer
book -- string - the book id/accronym e.g. BH

\end{fulllineitems}


\end{fulllineitems}



\section{CLiC KWICgrouper}
\label{apidoc:clic-kwicgrouper}\label{apidoc:module-kwicgrouper}\index{kwicgrouper (module)}
A module to look for patterns in concordances.
\index{Concordance (class in kwicgrouper)}

\begin{fulllineitems}
\phantomsection\label{apidoc:kwicgrouper.Concordance}\pysiglinewithargsret{\strong{class }\code{kwicgrouper.}\bfcode{Concordance}}{\emph{term}, \emph{text}, \emph{word\_boundaries=True}, \emph{length=50}, \emph{keep\_punctuation=True}, \emph{keep\_line\_breaks=False}}{}~\begin{quote}

This is a simple concordance for a text file. The input text
should a string that is cleaned, for instance:
\begin{quote}

text.replace(``
\end{quote}
\end{quote}

'', '' '').replace(''  '', '' '')
\begin{quote}

This function has two argument: the search term and the
text to be searched.

The length should be an integer
\end{quote}
\index{from\_multiple\_line\_file() (kwicgrouper.Concordance class method)}

\begin{fulllineitems}
\phantomsection\label{apidoc:kwicgrouper.Concordance.from_multiple_line_file}\pysiglinewithargsret{\strong{classmethod }\bfcode{from\_multiple\_line\_file}}{\emph{term}, \emph{input\_li}}{}
Construct a concordance that respect line breaks (rather than one
that treats the text as one large string)

TODO

\end{fulllineitems}

\index{list\_concordance() (kwicgrouper.Concordance method)}

\begin{fulllineitems}
\phantomsection\label{apidoc:kwicgrouper.Concordance.list_concordance}\pysiglinewithargsret{\bfcode{list\_concordance}}{}{}
List the actual concordance.

\end{fulllineitems}

\index{print\_concordance() (kwicgrouper.Concordance method)}

\begin{fulllineitems}
\phantomsection\label{apidoc:kwicgrouper.Concordance.print_concordance}\pysiglinewithargsret{\bfcode{print\_concordance}}{}{}
Print the lines of a concordance. For debugging purposes.

\end{fulllineitems}

\index{single\_line\_conc() (kwicgrouper.Concordance method)}

\begin{fulllineitems}
\phantomsection\label{apidoc:kwicgrouper.Concordance.single_line_conc}\pysiglinewithargsret{\bfcode{single\_line\_conc}}{}{}
Build a basic concordance based on a single string of text.

\end{fulllineitems}


\end{fulllineitems}

\index{KWICgrouper (class in kwicgrouper)}

\begin{fulllineitems}
\phantomsection\label{apidoc:kwicgrouper.KWICgrouper}\pysiglinewithargsret{\strong{class }\code{kwicgrouper.}\bfcode{KWICgrouper}}{\emph{concordance}}{}
This starts from a concordance and transforms it into a pandas dataframe
(here called textframe) that has five words to the left and right of the
search term in separate columns. These columns can then be searched for
and sorted.

Input:
A nested list of lists looking like:
\begin{quote}
\begin{description}
\item[{{[}}] \leavevmode
{[}'sessed of that very useful appendage  a `,
`voice',
`  for a much longer space of time than t'
{]},

\end{description}

...
\end{quote}

Each pattern needs \emph{its own} instantiation of the KWICgrouper object
because the self.textframe variable is changed in the filter method.
\index{args\_to\_dict() (kwicgrouper.KWICgrouper method)}

\begin{fulllineitems}
\phantomsection\label{apidoc:kwicgrouper.KWICgrouper.args_to_dict}\pysiglinewithargsret{\bfcode{args\_to\_dict}}{\emph{L5=None}, \emph{L4=None}, \emph{L3=None}, \emph{L2=None}, \emph{L1=None}, \emph{R1=None}, \emph{R2=None}, \emph{R3=None}, \emph{R4=None}, \emph{R5=None}}{}
Helper function to use
L1=''a'' type of functions

\end{fulllineitems}

\index{conc\_to\_df() (kwicgrouper.KWICgrouper method)}

\begin{fulllineitems}
\phantomsection\label{apidoc:kwicgrouper.KWICgrouper.conc_to_df}\pysiglinewithargsret{\bfcode{conc\_to\_df}}{}{}
Turns a list of dictionaries with L1-R5 values into a dataframe
which can be used as a kwicgrouper.

\end{fulllineitems}

\index{filter\_textframe() (kwicgrouper.KWICgrouper method)}

\begin{fulllineitems}
\phantomsection\label{apidoc:kwicgrouper.KWICgrouper.filter_textframe}\pysiglinewithargsret{\bfcode{filter\_textframe}}{\emph{kwdict}}{}
Construct a dataframe slice and selector on the fly.
This is no longer meta-programming as it does not use the eval
function anymore.

This returns None if there is no textframe

\end{fulllineitems}

\index{split\_nodes() (kwicgrouper.KWICgrouper method)}

\begin{fulllineitems}
\phantomsection\label{apidoc:kwicgrouper.KWICgrouper.split_nodes}\pysiglinewithargsret{\bfcode{split\_nodes}}{}{}
Splits the words into nodes that can be fed into a dataframe.

\end{fulllineitems}


\end{fulllineitems}

\index{clean\_punkt() (in module kwicgrouper)}

\begin{fulllineitems}
\phantomsection\label{apidoc:kwicgrouper.clean_punkt}\pysiglinewithargsret{\code{kwicgrouper.}\bfcode{clean\_punkt}}{\emph{text}}{}
Delete punktuation from a text.

Problem: turns CAN'T into CA NT

\end{fulllineitems}

\index{clean\_text() (in module kwicgrouper)}

\begin{fulllineitems}
\phantomsection\label{apidoc:kwicgrouper.clean_text}\pysiglinewithargsret{\code{kwicgrouper.}\bfcode{clean\_text}}{\emph{text}}{}~\begin{description}
\item[{Clean a text so that it can be used in a concordance. This includes:}] \leavevmode\begin{itemize}
\item {} 
all text to lowercase

\item {} 
deleting line-breaks

\item {} 
tokenizing and detokenizing

\end{itemize}

\end{description}

\end{fulllineitems}

\index{concordance\_for\_line\_by\_line\_file() (in module kwicgrouper)}

\begin{fulllineitems}
\phantomsection\label{apidoc:kwicgrouper.concordance_for_line_by_line_file}\pysiglinewithargsret{\code{kwicgrouper.}\bfcode{concordance\_for\_line\_by\_line\_file}}{\emph{input\_file}, \emph{term}}{}
Takes a file that has different line breaks that cannot be ignored
(for instance a file with a list of things) and makes it into a concordance

\end{fulllineitems}

\index{old\_clean\_punkt() (in module kwicgrouper)}

\begin{fulllineitems}
\phantomsection\label{apidoc:kwicgrouper.old_clean_punkt}\pysiglinewithargsret{\code{kwicgrouper.}\bfcode{old\_clean\_punkt}}{\emph{text}}{}
This ignores apostrophes and punctuation marks attached to the word
* an alternative way would be to replace-delete the punctuation from the text

\end{fulllineitems}



\section{CLiC Normalizer}
\label{apidoc:module-normalizer}\label{apidoc:clic-normalizer}\index{normalizer (module)}
Defines normalizers that can be used in the cheshire3 indexing workflow.


\section{CLiC Query Builder}
\label{apidoc:module-querybuilder}\label{apidoc:clic-query-builder}\index{querybuilder (module)}
Future module to handle the construction of cheshire3 CQL queries.


\section{CLiC Web app}
\label{apidoc:clic-web-app}

\subsection{Index}
\label{apidoc:index}\label{apidoc:module-web.index}\index{web.index (module)}
This is the most important file for the web app. It contains the various
routes that end users can use.

For instance

@app.route(`/about/', methods={[}'GET'{]})
def about():
\begin{quote}

return render\_template(``info/about.html'')
\end{quote}

Where /about/ is the link.


\subsection{API}
\label{apidoc:api}\label{apidoc:module-web.api}\index{web.api (module)}
This file is an extension of index.py. It generates the raw json API that
the keywords, cluster, and concordances use(d).

It needs to be refactored.


\subsection{Models}
\label{apidoc:models}\label{apidoc:module-web.models}\index{web.models (module)}
models.py defines the SQL tables that CLiC uses. These classes
provide a python interface to the SQL database so that you can write python
code that automatically queries the database.

This is heavily dependent on Flask-SQLAlchmey and SQLAlchemy


\chapter{Reporting issues}
\label{index:reporting-issues}

\chapter{Indices and tables}
\label{index:indices-and-tables}\begin{itemize}
\item {} 
\DUspan{xref,std,std-ref}{genindex}

\item {} 
\DUspan{xref,std,std-ref}{modindex}

\item {} 
\DUspan{xref,std,std-ref}{search}

\end{itemize}


\renewcommand{\indexname}{Python Module Index}
\begin{theindex}
\def\bigletter#1{{\Large\sffamily#1}\nopagebreak\vspace{1mm}}
\bigletter{c}
\item {\texttt{chapter\_repository}}, \pageref{apidoc:module-chapter_repository}
\item {\texttt{clusters}}, \pageref{apidoc:module-clusters}
\item {\texttt{concordance}}, \pageref{apidoc:module-concordance}
\indexspace
\bigletter{k}
\item {\texttt{keywords}}, \pageref{apidoc:module-keywords}
\item {\texttt{kwicgrouper}}, \pageref{apidoc:module-kwicgrouper}
\indexspace
\bigletter{n}
\item {\texttt{normalizer}}, \pageref{apidoc:module-normalizer}
\indexspace
\bigletter{q}
\item {\texttt{querybuilder}}, \pageref{apidoc:module-querybuilder}
\indexspace
\bigletter{w}
\item {\texttt{web.api}}, \pageref{apidoc:module-web.api}
\item {\texttt{web.index}}, \pageref{apidoc:module-web.index}
\item {\texttt{web.models}}, \pageref{apidoc:module-web.models}
\end{theindex}

\renewcommand{\indexname}{Index}
\printindex
\end{document}
